\chapter{The Finite element method, FEM}
\section{Problem definition}
\par We are interested in a solution of a partial differential equations of the type 
$$\mathcal{L}(\mathbf{u(x)}) = \mathbf{f(x)},~\forall \mathbf{x}\in\Omega$$
on a given domain $\Omega$, where $\mathcal{L}$ is a linear differential operator, 
${\mathbf{u} = \mathbf{u}(x_1,\ldots,x_n)=:\mathbf{u(x)},}$ 
is some tensor-valued function and 
${\mathbf{f} = \mathbf{f}(x_1,\ldots,x_n)=:\mathbf{f(x)}},$
tensor-valued right hand side.

\par It is necessary to impose boundary conditions on the boundary $\partial\Omega$ of the domain. These conditions are usually of type \textit{Dirichlet}, so that

$$ \mathbf{u(x)} = \mathbf{b_D(x)},~\forall \mathbf{x} \in \partial\Omega $$

where $\mathbf{b_D(x)}$ is some prescribed function. Another type of the boundary condition is so called \textit{Neumann}, where

$$ \nabla \mathbf{u(x)}\cdot\mathbf{n(x)} = \mathbf{b_N(x)},~\forall \mathbf{x}\in\partial\Omega $$

where $\mathbf{n(x)}$ is unit normal to the boundary.

\section{Weak solution and basis, variational formulation}

\par Yet, we didn't define function spaces for the functions in the problems like (PDE DEF). 
This is very important part and plays significant role in the finite element method.
\par Let us find such solutions to our problem, that all the desired functions $\mathbf{u}$ are in some space $\mathcal{S}$.
It is reasonable, to suppose, that the space is rich enough, to contain all the solutions, but the choice of this space is still up to us. 
\par We define the inner product of two functions well defined on $\Omega$ 

$$ \langle \mb f(\mb x),\mb g(\mb x) \rangle := \int_\Omega \mb f(\mb x) \mb g(\mb x) ~\mathrm{d\mathbf{x}} $$

and norm induced by the inner product

$$||\mb f(\mb x)|| := \sqrt{\langle \mb f(\mb x),\mb f(\mb x)\rangle}.$$

We say, that $\mathbf{u}$ is a \textbf{weak solution} to the problem (PDE), if
$$ \langle \mathcal{L}(\mathbf{u}(\mb x)) - \mathbf{f}(\mb x), \mathbf{s}(\mb x) \rangle = 0,~\forall \mathbf{s}\in\mathcal{S}. $$

\par Function $\mathbf{s}$ is often refered as a \textit{test function}. It is clear, that the space $\mathcal{S}$ is not of finite dimension. This is a very restrictive condition. 
One might try to find an approximation of a solution, $\mb{\tilde u}$ in a finite dimensional subspace, say $\mathcal{S}_n$, where $n \in \mathbb{N}_1$ is a dimension of this space. Let then $\{s_i\}, i=1,\ldots,n$
be the \textit{basis} of this space, so each function from our subspace $\mathcal{S}_n$ can be expressed as a linear combination of the basis functions
$$ \mathbf{\tilde u(\mb x)} = c_i\mathbf s_i(\mb x),~ \forall \mb{w} \in \mathcal{S}_h, $$
where summation convention is used.

\par The equation (PDE VAR) could be written in terms of the variational formulation. If we let
$$ L(\mb s(\mb x)) := \int_\Omega \mb s (\mb x) \mb f(\mb x) \mathrm d \mb x $$
and
$$ a(\mb u(\mb x), \mb s(\mb x)) := \int_\Omega (\mathcal L (\mb{\tilde u(\mb x)}) \mb s (\mb x) \mathrm d \mb x, $$
the problem (PDE) becomes an equality of the (uni)linear and bilinear form. The linearity of the forms is clear from the linearity of the Lebesgue integral. 

\section{Principles and algorithm}

\par We are thus interested in seeking a solutions of (PDE VAR). This can be rewritten taking $\mb s_j$ as the test function
$$ \langle \mathcal L (\mb{\tilde u(\mb x)}), \mb s_j(\mb x) \rangle = \langle \mb f(\mb x), \mb s_j(\mb x) \rangle $$
and decomposing approximate solution into our basis

\begin{align*}
\langle \mathcal L (c_i \mb s_i(\mb x)), \mb s_j(\mb x) \rangle &= \langle \mb f(\mb x), \mb s_j(\mb x) \rangle \\
c_i \langle \mathcal L (\mb s_i(\mb x)), \mb s_j(\mb x) \rangle &= \langle \mb f(\mb x), \mb s_j(\mb x) \rangle
\end{align*}
We let
$$ \mbb A := A_{ij} := \langle \mathcal L(\mb s_i(\mb x), \mb s_j(\mb x)) \rangle, $$
$$ \mb b := \langle \mb f(\mb x), \mb s_j(\mb x) \rangle $$
and 
$$ \mb c := \{c_i\} $$
set of the coefficients we are interested in. This is clearly a system of the equations known from linear algebra, $\mbb A \mb c = \mb b.$

\par We have derived the set of the equations that solves our problem in sense of a weak solution given by the condition (GALERKIN). 

\par Let suppose, for the sake of simplicity, that $\Omega \subset \mbb R^2$. 
It is evident, that we have chosen to find a solution of two dimensional spatial argument. This formalism
Integral over $\Omega$ induced by the inner product are decomposed into the sum of an integral over subdomains of $\Omega$. 
Such decomposition is done into triangles, e.g. a triangulation in 

\section{Finite element spaces}
