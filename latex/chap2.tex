\chapter{The Finite element method}

\par In the following sections we give a very brief introduction into the finite element method. 
We also refer more advanced reader who seeks more detail to [3 in LINDBO].

\section{Problem definition}

\par We are interested in a solution of a partial differential equations of the type 
$$\mathcal{L}(u(\mb x)) = f(\mb x),~\forall \mathbf{x}\in\Omega$$
on a given domain $\Omega$, where $\mathcal{L}$ is a linear differential operator, 
$u = u(x_1,\ldots,x_n)=: u(\mb x)$
and
$f = f(x_1,\ldots,x_n) =: f(\mb x)$
is some known right hand side.

\par It is necessary to impose boundary conditions on the boundary $\partial\Omega$ of the domain. These conditions are usually of type \textit{Dirichlet}, so that

$$ u = b_D(\mb x),~\forall \mathbf{x} \in \partial\Omega $$

where $b_D$ is a prescribed function. Another type of the boundary condition is so called \textit{Neumann}, where

$$ \nabla u \cdot \mb n(\mb x) = b_N(\mb x),~\forall \mathbf{x}\in\partial\Omega $$

where $\mathbf{n}$ is unit normal to the boundary.

\section{Weak solution and basis, variational formulation}

\par Yet, we didn't define function spaces for the functions in the problems like (PDE DEF). 
This is very important part and plays significant role in the finite element method.
\par Let us find such solutions to our problem, that the desired function $u$ is in some space $\mathcal{S}$.
It is reasonable, to suppose, that the space is rich enough, to contain all the solutions, but the choice of this space is still up to us. 
\par We define the inner product of two functions on $\Omega$ 

$$ (f(\mb x), g(\mb x)) := \int_\Omega fg ~\mathrm{d\mb x} $$

and norm induced by the inner product

$$||f|| := \sqrt{( f, f )}.$$

We say, that $u$ is a \textbf{weak solution} to the problem (PDE), if
$$ ( \mathcal{L}(u) - f, s(\mb x)) = 0,~\forall s \in \mathcal{S}. $$

\par Function $s$ is often refered as a \textit{test function}. It is clear, that the space $\mathcal{S}$ is not of finite dimension. This is a very restrictive condition. 
One might try to find an approximation of a solution, $\tilde u(\mb x)$ in a finite dimensional subspace, say $\mathcal{S}_n$, where $n \in \mathbb{N}_1$ is a dimension of this space. Let then $\{s_i(\mb x)\}, i=1,\ldots,n$
be the \textit{basis} of this space, so each function from our subspace $\mathcal{S}_n$ can be expressed as a linear combination of the basis functions
$$ \tilde u = c_i s_i,$$
where summation convention is used.

\par The equation (PDE VAR) could be written in terms of the variational formulation. If we let
$$ L(s) := \int_\Omega sf \mathrm d \mb x $$
and
$$ a(\tilde u, s) := \int_\Omega \mathcal L (\tilde u) s \mathrm d \mb x, $$
the problem (PDE) becomes an equality of the (uni)linear and bilinear form. The linearity of the forms is clear from the linearity of the Lebesgue integral. 

\section{Principles and algorithm}

\par We are thus interested in seeking a solutions of (PDE VAR). This can be rewritten taking $s_j(\mb x)$ as the test function
$$ ( \mathcal L (\tilde u), s_j ) = (f, s_j ) $$
and decomposing approximate solution into our basis
\begin{align*}
( \mathcal L (c_i s_i), s_j) &= (f, s_j), \\
c_i ( \mathcal L (s_i), s_j) &= 
\end{align*}
We let
$$ \mbb A := A_{ij} := (\mathcal L(s_i), s_j), $$
$$ \mb b := (f, s_j) $$
and 
$$ \mb c := \{c_i\} $$
set of the coefficients we are interested in. This is clearly a system of the equations known from linear algebra, $\mbb A \mb c = \mb b.$

\par We have derived the set of the equations that solves our problem in sense of a weak solution given by the condition (GALERKIN). 

\par Let suppose, for the sake of simplicity, that $\Omega \subset \mbb R^2$. 
Integral over $\Omega$ induced by the inner product is decomposed into the sum of integrals over subdomains of $\Omega$. 
In sense of FEM, such decomposition is done into triangles, e.g. a \textbf{triangulation} in $\mbb R^2$ into M cells.

We write
$$ \Omega =: \bigcup^M_{k=1} T_k, $$
so the matrix elements become
$$ A_{ij} = \int_\Omega \mathcal L(s_i)s_j \mathrm d \mb x = \sum_{k=1}^M \int_{T_k} \mathcal L(s_i)s_j \mathrm d \mb x.$$

\section{Finite element spaces}

\par The basis functions $s_i$ were not yet specified. Since the only restriction on these functions is, 
that $\{s_i\},~i=0,\ldots,n$ is the basis of our finite dimensional subspace $\mathcal S_n$, we choose them wisely. 

\par Because the system $\mbb A \mb c = \mb b$ will be solved, we would like them to vanish almost everywhere, i.e. to have non-zero value 
only on some element(triangle) with its neighbours. This implies, that the inner product $A_{ij} = (\mathcal L(s_i), s_j)$ forms a sparse matrix.

\par In the following, we refer to the \textbf{type} of the element. A type is simply a class of basis functions.
Most common choice of this class is so called \textit{Lagrange} polynomials.
\par The \textbf{order} is roughly the order of the interpolation polynomial.
\par The \textbf{shape} of the finite element is the geometry that defines the decomposition of $\Omega$.

\par For instance the finite element of type Lagrange, third order and triangular shape means triangulation of the $\Omega$ and  the choice of basis functions 
that are on each triangle Lagrange cubic polynomials.
 
